\documentclass{article}
\usepackage{amsfonts} 
\usepackage[english]{babel}
\newtheorem{axiom}{Axiom}[section]
\newtheorem{theorem}{Theorem}[section]
\newtheorem{definition}{Definition}[section]
\newtheorem{corollary}{Corollary}[theorem]
\newtheorem{lemma}[theorem]{Lemma}
\usepackage[letterpaper,top=2cm,bottom=2cm,left=3cm,right=3cm,marginparwidth=1.75cm]{geometry}

\begin{document}
\section{A Set of Axioms for the Real-Number System}

\subsection{The field axioms}

\begin{axiom}[Commutative laws]
\(x+y=y+x, \quad x y=y x .\)
\end{axiom}

\begin{axiom}[Associative laws]
\(x+(y+z)=(x+y)+z, \quad x(y z)=(x y) z .\)
\end{axiom}

\begin{axiom}[Distributive law]
\(x(y + z) = xy + xz.\)
\end{axiom}

\begin{axiom}[Existence of identity elements]
There exist two distinct real numbers, which we denote by 0 and 1, such that for every real \(x\) we have \(x + 0 = x\) and \(1 \cdot x = x\).
\end{axiom}

\begin{axiom}[Existence of identity elements]
For every real number \(x\) there is a real number \(y\) such that \(x + y = 0\).
\end{axiom}

\begin{axiom}[Existence of reciprocals]
For every real number \(x \neq 0\) there is a real number \(y\) such that \(xy = 1\).
\end{axiom}

\begin{theorem}[Cancellation law for addition]
if \(a + b = a + c\), then \(b = c\) (In particular, this shows that the number 0 of Axiom 1.4 is unique.)
\end{theorem}

\begin{theorem}[Possibility of subtraction]
Given \(a\) and \(b\), there is exactly one x such that \(a + x = b\). This \(x\) is denoted by \(b - a\). In particular, \(0 - a\) is written simply \(-a\) and is called the negative \(a\).
\end{theorem}

\begin{theorem}
\(b - a = b + (-a).\)
\end{theorem}

\begin{theorem}
\(-(-a) = a.\)
\end{theorem}

\begin{theorem}
\(a(b - c) = ab - ac.\)
\end{theorem}

\begin{theorem}
\(0 \cdot a = a \cdot 0 = 0\)
\end{theorem}

\begin{theorem}[Cancellation law for multiplication]
if \(ab = ac\) and \(a \neq 0\). then \(b = c\). (In particular, this shows that the number 1 of Axiom 1.4 is unique.)
\end{theorem}

\begin{theorem}[Possibility of division]
Given  \(a\) and \(b\) with \(a \neq 0\), there is exactly one \(x\) such that \(ax = b\). This \(x\) is denoted by \(b/a\) or \(\frac{b}{a}\) and is called the quotient of \(b\) and \(a\). In particular, \(1/a\) is also written \(a^{-1}\) and is called the reciprocal of \(a\).
\end{theorem}

\begin{theorem}
If \(a \neq 0\), then \(b/a = b \cdot a^{-1}\).
\end{theorem}

\begin{theorem}
If \(a \neq 0\), then \((a^{-1})^{-1}\) = a.
\end{theorem}

\begin{theorem}
If \(ab = 0\), then \(a = 0\) of \(b = 0\).
\end{theorem}

\begin{theorem}
\((-a)b = -(ab)\) and \((-a)(-b) = ab\).
\end{theorem}

\begin{theorem}
\((a/b) + (c/d) = (ad + bc)/(bd)\) if \(b \neq 0\) and \(d \neq 0\).
\end{theorem}

\begin{theorem}
\((a/b)(c/d) = (ac)/(bd)\) if \(b \neq 0\) and \(d \neq 0\).
\end{theorem}

\begin{theorem}
\((a/b)/(c/d) = (ad)/(bc)\) if \(b \neq 0\), \(c \neq 0\) and \(d \neq 0\).
\end{theorem}

\subsection{The order axioms}

\begin{axiom}
If \(x\) and \(y\) are in \(\mathbb{R}^{+}\), so are \(x + y\) and \(xy\). 
\end{axiom}

\begin{axiom}
For every real \(x \neq0\), either \(x \in \mathbb{R}^{+}\) or \(-x \in \mathbb{R}^{+}\), but not both.
\end{axiom}

\begin{axiom}
\(0 \notin \mathbb{R}^{+}\).
\end{axiom}

Now we can define the symbols $<$, $>$, $\leq$, and $\geq$, called, respectively, \emph{less than}, \emph{greater than}, \emph{less than or equal to}, and \emph{greater than or equal to}, as follows:


\begin{itemize}
\item \(x < y\) means that \(y - x\) is positive;

\item \(x > y\) means that \(x < y\);

\item \(x \leq y\) means that either \(x < y\) or \(x = y\);

\item \(y \geq x\) means that \(x \leq y\).
\end{itemize}

\begin{theorem}[Trichotomy law]
For arbitrary real numbers \(a\) and \(b\), exactly one of the three relations \(a < b\), \(b < a\), \(a = b\) holds.
\end{theorem}

\begin{theorem}[Transitive law]
If \(a < b\) and \(b < c\), then \(a < c\).
\end{theorem}

\begin{theorem}
If \(a < b\), then \(a + c < b + c\).
\end{theorem}

\begin{theorem}
If \(a < b\) and \(c > 0\), then \(ac < bc\).
\end{theorem}

\begin{theorem}
If \(a \neq 0\), then \(a^2 > 0\).
\end{theorem}

\begin{theorem}
\(1 > 0\).
\end{theorem}

\begin{theorem}
If \(a < b\) and \(c < 0\), then \(ac > bc\).
\end{theorem}

\begin{theorem}
If \(a < b\), then \(-a > -b\). In particular, if \(a < 0\), then \(-a > 0\).
\end{theorem}

\begin{theorem}
If \(ab > 0\), then both \(a\) and \(b\) are positive or both are negative.
\end{theorem}

\begin{theorem}
If \(a < c\) and \(b < d\), then \(a + b < c + d\).
\end{theorem}

\subsection{Integers and rational numbers}

\begin{definition}[Inductive set]
A set of real numbers is called an inductive set if it has the following two properties: \\
(a) the number 1 is in the set. \\
(b) For every \(x\) in the set, the number \(x + 1\) is also in the set.
\end{definition}

\begin{definition}[Positive integers]
A real number is called a positive integer if it belongs to every inductive set.
\end{definition}

\subsection{Upper bound of a set, maximum element, least upper bound (supremum)}

\begin{definition}[Least upper bound]
A number \(B\) is called a least upper bound of a nonempty set \(S\) if \(B\) has the following two properties: \\
(a) \(B\) is an upper bound for \(S\). \\
(b) No number less than \(B\) is an upper bound for \(S\).
\end{definition}

\begin{theorem}
Two different numbers cannot be least upper bounds for the same set.
\end{theorem}

\subsection{The least-upper-bound axiom (completeness axiom)}

\begin{axiom}
Every nonempty set \(S\) of real numbers which is bounded above has a supremum; that is, there is a real number \(B\) such that \(B = \sup S\).
\end{axiom}

\begin{theorem}
Every nonempty set \(S\) that is bounded below has a greatest lower bound; that is, there is a real number \(L\) such that \(S = \inf S\).
\end{theorem}

\subsection{The Archimedean property of the real-number system}

\begin{theorem}
The set \(\mathbb{P}\) of positive integers \(1, 2, 3, ...\) is unbounded above.
\end{theorem}

\begin{theorem}
For every real \(x\) there exists a positive integer \(n\) such that \(n > x\).
\end{theorem}

\begin{theorem}
If \(x > 0\) and if \(y\) is an arbitrary real number, there exists a positive integer \(n\) such that \(nx > y\).
\end{theorem}

\begin{theorem}
If three real numbers \(a\), \(x\), and \(y\) satisfy the inequalities
\[
a \leq x \leq a + \frac{y}{n}
\]
for every integer \(n \geq 1\), then \(x = a\).
\end{theorem}

\subsubsection{Fundamental properties of the supremum and infimum}

\begin{theorem}
Let \(h\) be given positive number and let \(S\) be a set of real numbers. \\
(a) If \(S\) has a supremum, then for some \(x\) in \(S\) we have
\[
x > \sup S - h.
\]
(b) If \(S\) has an infimum, then for some \(x\) in \(S\) we have
\[
x < \inf S + h.
\]
\end{theorem}

\begin{theorem}[Additive property]
Given nonempty subsets \(A\) and \(B\) of \(\mathbb{R}\), let \(C\) denote the set
\[
C = \{a + b \;|\; a \in A,\; b \in B\}.
\]
(a) if each of \(A\) and \(B\) has a supremum, then \(C\) has a supremum, and
\[
\sup C = \sup A + \sup B.
\]
(b) if each of \(A\) and \(B\) has an infimum, then \(C\) has an infimum, and
\[
\inf C = \inf A + \inf B.
\]
\end{theorem}

\begin{theorem}
Given two nonempty subsets \(S\) and \(T\) of \(\mathbb{R}\) such that
\[
s \leq t
\]
for every \(s\) in \(S\) and every \(t\) in \(T\). Then \(S\) has a supremum, and \(T\) has an infimum, and they satisfy the inequality
\[
\sup S \leq \inf T
\]
\end{theorem}

\subsection{Existence of square roots of nonnegative real numbers}

\begin{theorem}
    Every nonnegative real number \(a\) has a unique nonnegative square root.
\end{theorem}

Note: if \(a \geq 0\), we denote its nonnegative square root by \(a^{1/2}\) or by \(\sqrt{a}\). if \(a > 0\). the negative square root is \(-a^{1/2}\) or \(-\sqrt{a}\).

\section{Mathematical Induction, Summation Notation, and Related Topics}

\subsection{The principle of mathematical induction}

\emph{Method of proof by induction}. Let \(A(n)\) be an assertion involving an integer \(n\). We conclude that \(A(n)\) is true for every \(n \geq n_{1}\) if we can perform the following two steps:

(a) Prove that \(A(n_{1})\) is true.

(b) Let \(k\) be an arbitrary but fixed integer \(\geq n_{1}\). Assume that \(A(k)\) is true and prove that \(A(k + 1)\) is also true.

In actual practice \(n_{1}\) is usually \(1\). The logical justification for this method of proof is the following theorem about real numbers.

\begin{theorem}[Principle of mathematical induction]
    Let \(S\) be a set of positive integers which has the following two properties:

    (a) The number \(1\) is in the set \(S\).

    (b) If an integer \(k\) is in \(S\), then so is \(k + 1\).

    Then every positive integer is in the set \(S\).
\end{theorem}

\subsection{The well-ordering principle}

\begin{theorem}[Well-ordering principle]
    Every nonempty set of positive integers contains a smallest member.
\end{theorem}

\end{document}
